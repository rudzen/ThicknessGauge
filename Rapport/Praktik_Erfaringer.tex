\subsection{Indtryk og personlige konklusioner}
Set ud fra hele perioden, har det været givende at skulle drive et projekt frem, helt fra bunden. Den første dag havde jeg ikke den fjerneste ide hvor jeg skulle starte, og det endte med at der blev løst en hel del problematikker inden for optik og computer vision.
Derudover har det været godt at arbejde med et projekt hvor der ikke var nogen fornemmelse af man basalt set var "en-blandt-mange", i bedste robot/samlebånds ånd.


\section{Erfaringer}
Denne sektion indeholder information omkring de områder jeg mener har erhvervet mig mere erfaring.

\subsection{Computer Vision}
Projektet har stået på meget arbejde inden for dette område, og har lært rigtigt meget om emnet, både på det overfladiske niveau og i dybden. Der har været en del research af emnet generelt der har været meget givende, og var bærende i forhold til at løse opgaven.
Det blev nødvendigt at gå mere i dybden med nogle af de algoritmer der er benyttet, for bedre at kunne udnytte dem til projektets fordel.
Har førhen stiftet bekendtskab med billedbehandling i to kurser, \texttt{Brugerinteraktion og udvikling til mobile enheder} hvor  samt 4. semesters CDIO projekt hvor det blev brugt med meget positive resultater i begge dele.
Projektets process i hvorledes hovedopgaven er løst, er at finde i en semantisk beskrivelse i sektion \ref{ref::process}.

\subsection{C\texttt{++}}
Har intet fag haft for C\texttt{++}, men der har været en smule C i \texttt{Operativ Systemer} og \texttt{Parallelle Systemer}. Ellers er det nærmeste nok de indledende programmeringskurser på 1. og 2. semester. Jeg har selv også valgt før at benytte mig af C\texttt{++} i valgfaget \texttt{Kunstig intelligens i computerspil}, ved at konstruere en skakmotor med stor succes.
I løbet af hele projektet har jeg tilegnet mig meget viden omkring sproget i mange af dets facetter, ofte meget detaljeret.

\subsection{Versionsstyring \& Tests}
Softwaren, samt ekstra dokumentation er versionsstyret ved brug af git. Denne form for versionsstyring blev introduceret i 1. semesters kursus \texttt{Testmetoder og versionskontrol}. Hvilket også gælder for de unit tests softwaren indeholder.
