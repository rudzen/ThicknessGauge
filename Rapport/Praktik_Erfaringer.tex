\subsection{Indtryk og personlige konklusioner}
Set ud fra hele perioden, har det været fantastisk givende at skulle drive et projekt frem, helt fra bunden. Den første dag havde jeg ikke den fjerneste ide hvor jeg skulle starte, og det endte med at der blev løst en hel del problematikker inden for optik og computer vision.

\section{Erfaringer}
Denne sektion indeholder information omkring de områder jeg mener jeg har erhvervet mig mere erfaring.

\subsection{Computer Vision}
Projektet har stået på meget arbejde inden for dette område, og har lært rigtigt meget om emnet, både på det overfladiske niveau og i dybden. Der har været en del research af emnet generelt der har været meget givende, og var bærende i forhold til at løse opgaven. Det er altid nemt bare at læse et APIs dokumentation, og så benytte sig af det, men jeg var fast besluttet på at lære hvordan de rent faktisk virkede bagved. F.eks. en algoritme som Canny, hvilket er en edge detection algoritme, blev nøje studeret, hvilket gjorde mig i stand til at bedre kunne bruge denne og rent faktisk få et resultat ud der var positivt.
Har førhen stiftet bekendtskab med billedbehandling i to kurser, \texttt{Brugerinteraktion og udvikling til mobile enheder} samt 4. semesters CDIO projekt hvor det blev brugt med meget positive resultater i begge dele.
Projektets process i hvorledes hovedopgaven er løst, er at finde i en semantisk beskrivelse i sektion \ref{ref::process}.

\subsection{C++}
Har intet fag haft for C++, men der har været en smule C i \texttt{Operativ Systemer} og \texttt{Parallelle Systemer}. Ellers er det nærmeste nok de indledende programmeringskurser på 1. og 2. semester. Jeg har selv også valgt før at benytte mig af C++ i valgfaget \texttt{Kunstig intelligens i computerspil}, vil at konstruere en skakmotor med stor succes.
I løbet af hele projektet har jeg tilegnet mig meget viden omkring sproget i mange af dets facetter, ofte meget detaljeret.
