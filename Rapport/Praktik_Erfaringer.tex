\section{Indtryk og personlige konklusioner}
Det har været givende at skulle begynde på et projekt, hvor kun rammerne var fastlagt. Den første dag havde jeg ikke den fjerneste ide hvor jeg skulle starte, og projektet endte med, at der blev løst en hel del problematikker inden for optik og computer vision. Derudover har det været godt at arbejde på et projekt, hvor der blev sat pris på den individuelles evner og input. Der var ugentlige møder, hvor fremgangen blev gennemgået og problemer blev drøftet. 

\subsection{Kollegaer}
For projektet betød det temmelig meget, at arbejde sammen med gode personligheder og mennesker med mange forskellige evner, som ikke nødvendigvis var relateret til softwareudvikling. Det har specielt være givende at føre samtaler omkring emnet med ingeniører der har mere end 30 års erfaring inden for f.eks. optik og elektronik.

\section{Erfaringer}
Denne sektion indeholder information omkring de områder hvor det vurderes erfaring er blevet tilegnet og/eller udvidet.

\subsection{Computer Vision}
Projektet har stået på meget arbejde inden for dette område, og jeg har lært rigtigt meget om emnet, både på det overfladiske API benyttende niveau og i dybden. Der har været en del research om emnet generelt . Researchen var meget givende, og var bærende i forhold til at løse opgaven.
Det blev nødvendigt at gå mere i dybden med nogle af de algoritmer der er benyttet, for bedre at kunne udnytte dem til projektets fordel.\\
Jeg har førhen stiftet bekendtskab med billedbehandling i to kurser, 3. semesters\\ \texttt{62550 Brugerinteraktion og udvikling på mobile enheder}, for et selvvalgt projekt og i 4. semesters \texttt{02343 CDIO-Projekt}, hvor billedbehandling blev brugt med meget positive resultater i begge fag.
Behandlingen af de informationer fra billederne, har svagt berørt elementer der har været oppe i \texttt{02323 Introduktion til statistik}.
Projektets process i hvorledes hovedopgaven er løst, kan findes i en semantisk beskrivelse i sektion \ref{ref::process}.

\subsection{C\texttt{++}}
Jeg har intet fag haft om C\texttt{++}, men der har været en smule C i \texttt{02349 Operativsystemer} og \texttt{02347 Parallelle systemer}. Ellers er de nærmeste relaterede nok de to kurser i Java på 1. og 2. semester \texttt{02312 Indledende programmering} samt \texttt{02324 Videregående programmering}.
Jeg havde dog selv valgt at benytte mig af C\texttt{++} i valgfaget \texttt{62598 Kunstig intelligens i computerspil} (E2016), ved at konstruere en skakmotor med stor succes.
I løbet af hele projektet har jeg tilegnet mig meget viden omkring sproget i mange af dets facetter, ofte meget detaljeret. Efter dette projekt er det min klare holdning at C\texttt{++} indeholder utroligt meget dybde, og det står klart for mig, at det er et af de mest komplekse, og kraftfulde sprog der findes.

\subsection{Versionsstyring \& Tests}
Softwaren, samt ekstra dokumentation er versionsstyret ved brug af git. Denne form for versionsstyring blev introduceret i 1. semesters kursus \texttt{02315 Versionsstyring og testmetoder}, hvilket også gælder for de unit tests softwaren indeholder.

\subsection{Virkeligheden snyder}
Jeg har lært en utrolig vigtig ting, ved at arbejde med virkelighedsforhold, som f.eks. lys og/eller farver (intensiteter). Når noget på papiret virker som en god løsning, skal det altid gennemtestes og sættes under stress.
Det viser sig ofte at noget der virker den ene dag, ikke virker den næste, og det kan typisk være noget så banalt som f.eks. en ændring af lysforhold pga. vejret.

