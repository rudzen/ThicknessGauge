\section{Indtryk og personlige konklusioner}

Set ud fra hele perioden, har det været givende at skulle drive et projekt frem, helt fra bunden. Den første dag havde jeg ikke den fjerneste ide hvor jeg skulle starte, og det endte med at der blev løst en hel del problematikker inden for optik og computer vision.
Derudover har det været godt at arbejde med et projekt, hvor der ikke var nogen fornemmelse af man basalt set var "en-blandt-mange", i bedste robot/samlebånds ånd. Den form for udvikling hæmmer kreativiteten hos nye udviklere, og presser mennesker ind i positioner hvor der ofte opstår stressfyldte situationer.

\subsection{Kollegaer}
De kollegaer jeg har arbejdet med har vist sig ikke alene at være dygtige, men også erfarne. Selvom de fleste af dem ikke er relateret til software udvikling, betyder det at have gode personligheder og mennesker med mange forskellige evner temmelig meget. Det har specielt være givende at føre samtaler omkring emnet med ingeniører der har mere end 30 års erfaring inden for f.eks. optik og elektronik.

\section{Erfaringer}
Denne sektion indeholder information omkring de områder hvor det vurderes erfaring er blevet tilegnet.

\subsection{Computer Vision}
Projektet har stået på meget arbejde inden for dette område, og har lært rigtigt meget om emnet, både på det overfladiske APX benyttende niveau og i dybden. Der har været en del research af emnet generelt der har været meget givende, og var bærende i forhold til at løse opgaven.
Det blev nødvendigt at gå mere i dybden med nogle af de algoritmer der er benyttet, for bedre at kunne udnytte dem til projektets fordel.
Har førhen stiftet bekendtskab med billedbehandling i to kurser, 3. semesters \texttt{62550 Brugerinteraktion og udvikling på mobile enheder}, for et selvvalgt projekt og i 4. semesters \texttt{02343 CDIO-Projekt}, hvor det blev brugt med meget positive resultater i begge dele.
Behandlingen af de informationer fra billederne, har svagt berørt elementer der har været oppe i \texttt{02323 Introduktion til statistik}.
Projektets process i hvorledes hovedopgaven er løst, er at finde i en semantisk beskrivelse i sektion \ref{ref::process}.

\subsection{C\texttt{++}}
Har intet fag haft for C\texttt{++}, men der har været en smule C i \texttt{02349 Operativsystemer} og \texttt{02347 Parallelle systemer}. Ellers er det nærmeste nok de to kurser i Java på 1. og 2. semester \texttt{02312 Indledende programmering} samt \texttt{02324 Videregående programmering}.
Jeg har selv også valgt før at benytte mig af C\texttt{++} i valgfaget \texttt{62598 Kunstig intelligens i computerspil}, ved at konstruere en skakmotor med stor succes.
I løbet af hele projektet har jeg tilegnet mig meget viden omkring sproget i mange af dets facetter, ofte meget detaljeret. Efter dette projekt er det min klare holdning at C\texttt{++} indeholder utroligt meget dybde, og det står klart at det er et af de mest komplekse, og kraftfulde sprog der findes.

\subsection{Versionsstyring \& Tests}
Softwaren, samt ekstra dokumentation er versionsstyret ved brug af git. Denne form for versionsstyring blev introduceret i 1. semesters kursus \texttt{02315 Versionsstyring og testmetoder}. Hvilket også gælder for de unit tests softwaren indeholder.

\subsection{Virkeligheden snyder}
Ved at arbejde med virkelighedsforhold, som f.eks. lys og/eller farver (intensiteter), har jeg lært en utrolig vigtig ting. Det at noget på papiret virker som en god løsning, skal altid gennemtestes og sættes under stress.
Det viser sig ofte at noget der virker den ene dag, virker ikke den næste, og det kan typisk være noget så banalt som f.eks. det er blevet overskyet.
