\section{Kommandolinje Interface}
Softwares understøtter styring via et hertil designet kommandolinje interface. Denne sektion beskriver de forskellige muligheder stillet til rådighed ud fra det.

Fungerer ved at starte programmet med de valgte parametrer, som f.eks.\\\\
\texttt{\textbf{ThicknessGauge.exe -d -{}-glob\_name=camera -{}-show\_windows=0}}
\\\\
For en liste over de muligheder der er til rådighed, kan programmet startes med:\\\\
\texttt{\textbf{ThicknessGauge.exe -{}-help}}\\


\subsection{Kommandoerne}

* denoterer ikke komplet understøttet, eller ikke komplet i softwarens nuværende tilstand.

\begin{table}[h]
	\centering
	\label{kommandolinjekontakter}
	\begin{tabular}{lll}
	-d	& -{}-demo & Kører softwaren i dens regulære tilstand \\
	-c	& -{}-calibrate & Kører kamera kalibrering  \\
	-t*	& -{}-test*  &  Kører test tilstand \\
	-g*  & -{}-save\_glob* & Kører en prædefineret opsætning og gemmer en billedserie.
	\end{tabular}
	\caption{Kommandolinje kontakter}
\end{table}

\begin{table}[h]
	\centering
	\label{kommandolinjeindstilling}
	\begin{tabular}{ll}
		-{}-phase\_two\_exposure* & Tvinger programmet til at bruge en eksponering i fase to. \\
-z* & Tvinger aflæsning af område, til nulmålinger. \\
-{}-glob\_name & Brug gemt billedserie, eller "camera" for live feed. \\
-{}-opencv\_threads & Tvinger OpenCV til at bruge N antal tråde. \\
-{}-calibrate\_output & Filnavn til at gemme kamera kalibrering under. \\
-{}-camera\_settings & Kamera kalibrerings fil der skal bruges under processen. \\
-{}-record\_video* & Opretter en video fil af processen. \\
-{}-show\_windows & Optegner processen visuelt. (Giver ikke nøjagtige målinger.) \\
-{}-frames* & Sætter antal frames der skal benyttes under kalkuleringerne. \\
-{}-test\_suite* & Filnavn for at gemme test sæt under. \\
-{}-version & Udskriver programmets version. \\
-{}-help & Viser komplet hjælp af alle kommandolinje parametrer. \\
	\end{tabular}
	\caption{Kommandolinje: indstillinger}
\end{table}
