%praktik konklusion
\section{Praktik Resumé}
Gennem hele perioden arbejdede jeg selvstændigt, med ugentlige statusmøder.
De ugentlige statusmøder bestod primært af fremgangsfremlæggelse fra min side, og generel udveksling af idéer fra alle parter tilstede.
Ud over min projektvejleder hos DELTA, deltog en udvikler og deres førende r\&d ingeniør. Det resulterede oftest i at disse møder blev givende,
i form af nye idéer og potentielle løsninger til problemer, der kunne opstå grundet projektets natur. Jeg fik løst problemet omkring opgaven, og
der var generelt en stor tilfredshed i hele perioden fra DELTAs side omkring projektet og dets fremgang. 


\section{Praktik Konklusion}
Havde originalt søgt en anden praktik position hos DELTA, men blev tilbudt at arbejde på et andet projekt, fordi den stilling jeg søgte allerede
var udfyldt. Fik dog mulighed for at møde vejlederen og afdelingens udviklere, inden jeg skulle tage stilling til om jeg ville acceptere eller ej.
Set i bagklogskabens lys er jeg glad for at jeg accepterede, fordi det gav mig mulighed for at arbejde på dette projekt, helt fra bunden af.
Den originale praktikstilling jeg havde søgt, vurderede jeg til at være ca. 4-6 ugers arbejde, og det ville reelt være triviel repetition af viden jeg
allerede besad.
I det hele taget er jeg godt tilfreds med praktikopholdet hos DELTA, og mener det har udvidet min viden omkring forskellige teknologier.
Jeg kan kategorisere den viden jeg har tilegnet/forbedret for mit vedkomme i to lejre, generel- og specialviden. Den generelle viden er almindelig viden
der er bredt, som f.eks. programmeringssprog, brug af værktøjer osv. På den anden side er der viden omkring billedbehandling, diverse algoritmer i denne kategori
og almindelig optik/kamera viden. Denne specialviden tilføjer til mit repertoire der understøtter den generelle viden.
Til trods for den store mængde teori og praksis jeg har tilegnet mig inden for computer vision området, er dette ikke et område, givet muligheden, jeg ønsker at
arbejde inden for. Dog er denne viden stadig virkelig god at have, da der kan opstå situationer hvor det virkelig kan blive brugbart.
Det er altid en chance at tage når man skal arbejde med et afklaringsprojekt, da i tilfælde af problemet eller opgaven ikke kan løses, vil kunne lede til et tab af selvværd fra
en praktikants side. DELTA havde ikke nogen store forventninger til, at projektet rent faktisk kunne løses med de midler til rådighed, fordi de godt vidste hvor
mange problematikker der kan opstå i den type projekter. Det var netop også årsagen til at det var et studenterprojekt, prisen.


\section{Praktik samlet vurdering af DELTA}
Kan derfor konkludere at DELTA, for mig, har været et rigtigt godt praktiksted, og kan anbefale andre at overveje det, hvis de vel og mærke er klar over at hos DELTA, for man ikke noget forærende. Man skal selv være fremme i skoene, og det er en del af opgaven.
Min helhedsvurdering af DELTA, er derfor 5.
