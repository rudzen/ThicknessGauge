\documentclass[11pt,a4paper,danish]{article}
\usepackage[utf8]{inputenc}
\usepackage[danish]{babel} % danske overskrifter
\usepackage[T1]{fontenc} % fonte (output)
\usepackage{graphicx} % indsættelse af billeder
\usepackage{caption}
\usepackage{subcaption}
\usepackage{wrapfig}
\usepackage{geometry}
\usepackage{titlesec}
\usepackage{fancyhdr}
\usepackage{lastpage}
%\usepackage{biblatex}
\usepackage{hyperref}
\usepackage{gensymb}

\usepackage{setspace}
\usepackage{tocloft}
\usepackage{parskip}

\usepackage[nottoc]{tocbibind}

%\usepackage{subfig}
%\usepackage{inconsolata}
%\usepackage{enumitem}

\titleformat{\paragraph}[runin]
{\bfseries\scshape}{\theparagraph}{1em}{}

% BEGIN JAVA CODE SECTION
\usepackage{listings}
\usepackage{color}

\definecolor{dkgreen}{rgb}{0,0.6,0}
\definecolor{gray}{rgb}{0.5,0.5,0.5}
\definecolor{mauve}{rgb}{0.58,0,0.82}

\lstset{frame=none,
  language=Java,
  aboveskip=3mm,
  belowskip=3mm,
  showstringspaces=false,
  columns=flexible,
  basicstyle={\small\ttfamily},
  numbers=none,
  numberstyle=\tiny\color{gray},
  keywordstyle=\color{blue},
  commentstyle=\color{dkgreen},
  stringstyle=\color{mauve},
  breaklines=true,
  breakatwhitespace=true,
  tabsize=2
}
% END JAVA CODE SECTION

% BEGIN C++ code
\definecolor{listinggray}{gray}{0.9}
\definecolor{lbcolor}{rgb}{0.9,0.9,0.9}
\lstset{
backgroundcolor=\color{lbcolor},
    tabsize=4,
%   rulecolor=,
    language=[GNU]C++,
        basicstyle=\scriptsize,
        upquote=true,
        aboveskip={1.5\baselineskip},
        columns=fixed,
        showstringspaces=false,
        extendedchars=false,
        breaklines=true,
        prebreak = \raisebox{0ex}[0ex][0ex]{\ensuremath{\hookleftarrow}},
        frame=single,
        numbers=left,
        showtabs=false,
        showspaces=false,
        showstringspaces=false,
        identifierstyle=\ttfamily,
        keywordstyle=\color[rgb]{0,0,1},
        commentstyle=\color[rgb]{0.026,0.112,0.095},
        stringstyle=\color[rgb]{0.627,0.126,0.941},
        numberstyle=\color[rgb]{0.205, 0.142, 0.73},
%        \lstdefinestyle{C++}{language=C++,style=numbers}’.
}
\lstset{
    backgroundcolor=\color{lbcolor},
    tabsize=4,
  language=C++,
  captionpos=b,
  tabsize=3,
  frame=lines,
  numbers=left,
  numberstyle=\tiny,
  numbersep=5pt,
  breaklines=true,
  showstringspaces=false,
  basicstyle=\footnotesize,
%  identifierstyle=\color{magenta},
  keywordstyle=\color[rgb]{0,0,1},
  commentstyle=\color{Darkgreen},
  stringstyle=\color{red}
  }
%END C++ code



%\setcounter{secnumdepth}{3}

\geometry{
 a4paper,
 total={210mm,297mm},
 left=20mm,
 right=20mm,
 top=40mm,
 bottom=20mm,
 }

\pagestyle{fancy}
\fancyhf{}
\rhead{Delta, \date{\today}}
\lhead{Rudy Alex Kohn}
\chead{Uge 2}
\cfoot{Side \thepage \hspace{1pt} af \pageref{LastPage}}

% TOC som links
\usepackage{hyperref}
\hypersetup{
    colorlinks,
    citecolor=black,
    filecolor=black,
    linkcolor=black,
    urlcolor=blue
}

\title{Stripetykkelsesmåler\\Afsluttende Rapport For Praktikophold}
\author{Rudy Alex Kohn}
\date{8. Juni 2017}

\setlength{\parindent}{0pt}

\begin{document}

\maketitle
\renewcommand{\contentsname}{Indeks}
\renewcommand{\listfigurename}{Figurliste}
\renewcommand{\figurename}{Figur}
\renewcommand\refname{Referencer}
%PWNED! :> nice!

\begin{center}
Rudy Alex Kohn\\
ruak@force.dk\\
s133235@student.dtu.dk
\end{center}

\vspace{25mm}

\begin{center}
\includegraphics[scale=0.3]{Billeder/kunlogo.png}
\includegraphics[scale=0.15]{Billeder/DELTA_1024px.png}
\end{center}

\newpage
\setlength\cftparskip{-2pt}
%\setlength\cftbeforechapskip{0.0pt}\singlespacing
\tableofcontents
%\singlespacing

\listoffigures
\listoftables

%1:  produktspecifikation. forudsætninger
%Performance  tykkelsesmålinh præsisosn +/- 0.1 mm
%Product cist max 25,000,- kr gerne mindre.
%2 teknologier og teknologi valg – begrundelse
%3 forsøgsopstilling – gennemgang.
%4 databehandling  procesdiagram.
%5 resulateter  - hvordan ser det ud.
%6 Videre arbejde- plan for den sidste del af projektet- hvor langt når vi, dokumentation etc.

\section{Introduktion}

Denne rapport afdækker praktikforløbet hos Delta (Force, Hørsholm). Den beskriver de forskellige faser i projektet, samt belyser de problematikker der opstod i forbindelse med at løse opgaven.
Praktikopholdet hos Delta er baseret på et afklaringsprojekt med henblik på at få afdækket om hvorvidt det er muligt at kunne måle højden af en vejstribe (termoplast). I den forbindelse var der fra starten en del problematik, bl.a. var der på forhånd ikke udlagt hvorledes dette skulle foregå. På den baggrund startede forløbet med at der blev brugt tre uger på indsamling af information omkring hvilken form for teknologi der skulle benyttes. Valget endte med at stå mellem 3d laserskanning eller computer vision.

Det viste sig dog hurtigt at laserskanning var udelukket grundet de høje omkostninger det ville indebære, og der blev derfor fundet et ældre kamera frem der kunne benyttes til projektet.



\section{Produktets specifikationer}

Krav for at produktet rent faktisk kan blive en realitet.

\begin{itemize}
    \item Processen må samlet set ikke tage særlig lang tid, gerne under 4 sekunder.
    \item Målingens præcision skal være på $\pm 0.1$ mm
    \item Omkostningerne skal holde sig under 25.000,- DKr.
\end{itemize}

\section{Teknologivalg}
Denne sektion beskriver de forskellige delelementer projektet består af. Det er både med hensyn til hardware men hvilke udviklingsværktøj der blev benyttet. For at kunne danne et komplet billede, er der information medtaget omkring nogle af beslutninger og deres bevæggrunde. 

\subsection{Kamera}
Der er valgt at benytte sig af et kamera, der på baggrund af en laser der kaster en linje, kan måle forskel i højden fra hvor laserens linje kastes på en termoplastisk markering i forhold til, hvor den rammer ved siden af markeringen.

\subsection{Laser}
Laseren kaster en linje ned på markeringsområdet med en styrke der gør det er muligt at kunne aflæse dens lokalisation.

\subsection{OpenCV}
Til denne proces bliver Open Computer Vision \cite{OpenCV} benyttet, hvilket er et gratis computer vision system der gør det muligt at lave behandling på dataen for at lokalisere laserlinjen på de billeder kameraet tager.

\subsection{C++}
Udviklingen har indtil videre foregået i C++, da dette potentielt åbner op for at benytte sig af softwaren, eller dele af den, på generisk vis på tværs af diverse platforme, da softwaren er skrevet med udgangspunkt i standard template library.
Softwaren kræver en compiler der understøtter C++14, for en komplet liste over understøttede compilere, se \href{http://en.cppreference.com/w/cpp/compiler_support}{her}.

\subsection{Begrundelse}
Valget af kamera og laser er baseret på prisen, et moderne kamera med en tilstrækkelig kvalitet koster under 6.000,- DKr, og en linjelaser koster mellem 2.000,- DKr. og 3.000,- DKr. En laser scannings enhed koster over 15.000,- Dkr. hvilket vil sætte for meget pres på budgettet.

Open Computer Vision er gratis, og kan benyttes i ubegrænset omfang, også i kommercielle forbindelser.



\section{Forsøgsopstillingen}
Denne sektion omhandler opstillingen der er blevet benyttet til at lave test målinger med.



\include{4_Databehandlingen}

\section{Nuværende resultater}

Denne sektion beskriver de resultater, både de endelige samt metoder og algoritmer brugt og udviklet under forløbet.

\subsection{Antal billeder pr. måling}

Grundet fluktuerende intensitet omgivet laseren, er det nødvendig at udføre hver enkelt måling over en billedserie.

\subsection{Linearitet}

Ud fra en række målinger i serie, dvs. i højder der spænder fra 1mm til 6mm, er der blevet opsat en test model for hvorvidt der er linearitet.


(indsæt tabel her)


Den ovenstående tabel viser de målinger der er foretaget i forskellige højder. For at sikre et rimeligt spænd er hver enkelt måling bestående af 25 målinger, hvor de hver er foretaget over 25 billeder.

Ud fra de ovenstående tal, viser det sig at der rent faktisk er et rimelig god linearitet i målingerne.
Dette afgøres ved $R^2$ værdien der indikerer lineariteten, og en værdi på $1.00$ er perfekt og vil kun opstå ved fuldstændig uniform data.

Disse målinger udgør kalibreringen, dvs. reference.

\subsubsection{Verificering af kalibreringen}

Ved at regne baglæns med de givne informationer, er det muligt at lokalisere anormaliteter i de enkelte højder.
Den angivne linjeformel er

\begin{equation}
content...
\end{equation}

Indsætning af de resulterede data i denne formel giver følgende

(data her)


Diskrepansen for selve målingerne findes ved (i promille)

(formel her)


Dette frembringer følgende resultater, hvor det er tydeligt de ligger ret tæt på selve målingerne.



\subsubsection{Ny måling}

Verificering af kalibreringsmålingen er som udgangspunkt kun god til at tjekke om de informationer man har til rådighed er fornuftige.

25 nye målinger ser således ud

Ved indsætning i kalibrerings formlerne fås således

(indsæt resultat her)




\section{Videre arbejde}

Denne sektion afdækker områder hvorpå der kan arbejdes videre med i projektet.

\subsection{Bedre opstilling}

Den nuværende opstilling er primitiv, dette gør den er meget usikker og følsom. Forbedringer på dette punkt vil være en stor fordel. Hvorledes dette realiseres, er afhængig af hvilken retning projektet drejer, og ligger på nuværende tidspunkt kun grund til antagelser bedst udeladt.

\subsection{Sammenhæng af laser og kamera}

De to hovedelementer i opstillingen, kameraet og laseren, er separeret på nuværende tidspunkt. Dette gør det meget svært at lave vinklede målinger, da flytningen af den ene kræver at den anden flyttet i samme grad, hvilket er et stort problem at gøre med den påkrævede nøjagtighed individuelt.


%\bibliographystyle{unsrt}
%\bibliography{ref}

\end{document}
