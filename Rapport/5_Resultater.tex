\section{Indledende undersøgelser}
De første tre uger blev stort set kun brugt på at søge efter information omkring hvordan det var muligt at skulle kunne få noget information der kunne bruges til noget. Der blev primært kigget efter 3d laser scannings enheder, da det ville være det mest præcise og kræve mindst i opsætning og udvikling osv. På trods af at budgettet var på 25.000,- DKr, blev det i sidste ende vurderet for dyrt med en laser scannings enhed til omkring 18.000,-. Det virkede i det hele taget ikke som om der var nogen der havde skænket projektet en tanke på noget område før det startede.
Derfor blev der fundet et næsten 10 år gammelt kamera frem der kunne bruges, og der blev bestilt en laser i stor hast der kunne bruges til formålet.
4. uge kom der så gang i udviklingen, hvilket betød at ud over de alm. børnesygdomme grundet konfigurationen af kameraet, også skulle bruges et API til kameraet. Desværre viste det sig at deres API ikke virkede når det kom til at få informationer fra kameraet, som f.eks. dets indstillinger og/eller billeder.
Derefter gik der så en uge med at få sat en OpenCV op til at benytte det ældre API, konfigurere det hele og få hul igennem til kameraet via C++.


\section{Nuværende resultater}

Denne sektion beskriver de resultater, både de endelige samt metoder og algoritmer brugt og udviklet under forløbet.

\subsection{Antal billeder pr. måling}

Grundet fluktuerende intensitet omgivet laseren, er det nødvendig at udføre hver enkelt måling over en billedserie.

\subsection{Linearitet}

Ud fra en række målinger i serie, dvs. i højder der spænder fra 1mm til 6mm, er der blevet opsat en test model for hvorvidt der er linearitet.


(indsæt tabel her)


Den ovenstående tabel viser de målinger der er foretaget i forskellige højder. For at sikre et rimeligt spænd er hver enkelt måling bestående af 25 målinger, hvor de hver er foretaget over 25 billeder.

Ud fra de ovenstående tal, viser det sig at der rent faktisk er et rimelig god linearitet i målingerne.
Dette afgøres ved $R^2$ værdien der indikerer lineariteten, og en værdi på $1.00$ er perfekt og vil kun opstå ved fuldstændig uniform data.

Disse målinger udgør kalibreringen, dvs. reference.

\newpage

\subsubsection{Verificering af kalibreringen}

Ved at regne baglæns med de givne informationer, er det muligt at lokalisere anormaliteter i de enkelte højder.
Resultatet angiver hvor præcis lineariteten rent faktisk er, da resultaterne gerne skulle svare til højden i millimeter, der i dette tilfælde er.

Den angivne linjeformel er

\begin{equation}
px2mm = \frac{avg - b}{a}
\end{equation}

Indsætning af de resulterede data i denne formel giver følgende

(data her)


Diskrepansen for selve målingerne findes ved (i promille)

\begin{equation}
dev\permil =  \vert px2mm - mm \vert \cdot 1000
\end{equation}
(formel her)


Dette frembringer følgende resultater, hvor det er tydeligt de ligger ret tæt på selve målingerne.



\subsubsection{Ny måling}

Verificering af kalibreringsmålingen er som udgangspunkt kun god til at tjekke om de informationer man har til rådighed er fornuftige.

25 nye målinger ser således ud

Ved indsætning i kalibrerings formlerne fås således

(indsæt resultat her)


