\section{Nuværende resultater}

Denne sektion beskriver de resultater, både de endelige samt metoder og algoritmer brugt og udviklet under forløbet.

\subsection{Antal billeder pr. måling}

Grundet fluktuerende intensitet omgivet laseren, er det nødvendig at udføre hver enkelt måling over en billedserie.

\subsection{Linearitet}

Ud fra en række målinger i serie, dvs. i højder der spænder fra 1mm til 6mm, er der blevet opsat en test model for hvorvidt der er linearitet.


(indsæt tabel her)


Den ovenstående tabel viser de målinger der er foretaget i forskellige højder. For at sikre et rimeligt spænd er hver enkelt måling bestående af 25 målinger, hvor de hver er foretaget over 25 billeder.

Ud fra de ovenstående tal, viser det sig at der rent faktisk er et rimelig god linearitet i målingerne.
Dette afgøres ved $R^2$ værdien der indikerer lineariteten, og en værdi på $1.00$ er perfekt og vil kun opstå ved fuldstændig uniform data.

Disse målinger udgør kalibreringen, dvs. reference.

\subsubsection{Verificering af kalibreringen}

Ved at regne baglæns med de givne informationer, er det muligt at lokalisere anormaliteter i de enkelte højder.
Den angivne linjeformel er

\begin{equation}
content...
\end{equation}

Indsætning af de resulterede data i denne formel giver følgende

(data her)


Diskrepansen for selve målingerne findes ved (i promille)

(formel her)


Dette frembringer følgende resultater, hvor det er tydeligt de ligger ret tæt på selve målingerne.



\subsubsection{Ny måling}

Verificering af kalibreringsmålingen er som udgangspunkt kun god til at tjekke om de informationer man har til rådighed er fornuftige.

25 nye målinger ser således ud

Ved indsætning i kalibrerings formlerne fås således

(indsæt resultat her)


