%PraktikDelta
\section{DELTA}

\subsection{Lidt om}
DELTA Road Sensors er afdelingen hvor der udvikles produkter der er vejrelaterede. F.eks. skiltemåler, vejstribe RL målere, som er både håndholdt og mobile (hvor mobile også kan betyde påsat et køretøj). Der eksisterer kunder verden over, og det er typisk instanser der tilsvarer det danske vejdirektorat der er kunder.

\subsection{Afdelingen}
Da både hardware- og sofwaredele bliver udviklet helt fra bunden, spiller softwareudviklerne en stor rolle i hele processen. Det reflekteres i deres konstante involvering i processen samt at de typisk skal have et bredt kendskab til produktet. Grundet denne viden udøver softwareudviklerne også support på produkterne, hvilket er en meget vigtig del af det samlede salg, da nogle af produkter ligger på omkring en million kroner.

\subsection{Udviklingsmetoderne}
På afdelingen findes der ingen forskrifter til hvorledes produkterne udvikles og hvilke værktøjer der skal bruges. Softwareudviklerne besidder fuld kontrol over hvorledes udviklingsprocessen skal foregå, hvilket giver en høj grad af fleksibilitet, men kræver som udgangspunkt, at hver udvikler er kendt med alle de gængse metoder og sprog.

\subsection{Samarbejde med andre afdelinger}
Afdelingerne fungerer som enkelte instanser indenfor virksomheden.
Det resulterer derfor i at de forskellige afdelinger registrerer deres tidsforbrug. På den måde kan afdelingerne holde styr på deres budgetforbrug, internt og eksternt. Det er DELTAs politik at budgettere korrekt og nøjagtig, også når afdelingerne benytter sig af hinandens ekspertiser.  




